\section{Methods}
\label{sec:methods}

\subsection{Experimental Design}
\label{subsec:experimental_design}

Our experimental paradigm centers on poker deception detection as a naturalistic test of Theory of Mind capabilities. Unlike traditional false-belief tasks that may be solved through pattern matching, poker scenarios require models to reason about opponent mental states, beliefs, and intentions in dynamic, strategically complex situations.

\subsubsection{Stimulus Construction}

We developed 30 poker scenarios (15 pairs) following a controlled design:

\begin{itemize}
    \item \textbf{Scenario Structure}: Each scenario presents a poker hand history including hero cards, board texture, pot size, and opponent action, followed by contextual information about opponent psychology and playing style.
    
    \item \textbf{Binary Classification}: Models predict whether the opponent's bet represents a BLUFF (deceptive intent) or VALUE bet (sincere intent to extract value from a strong hand).
    
    \item \textbf{Controlled Pairing}: Each base scenario has both bluff and value variants, keeping poker mechanics identical while varying only opponent psychological profiles.
    
    \item \textbf{Contextual Richness}: Opponent descriptions include behavioral patterns, historical tendencies, and psychological profiles designed to elicit genuine mental state reasoning.
\end{itemize}

For example, Scenario S1 presents identical hand strength and bet sizing but contrasts an "aggressive player who triple-barrels missed draws" (bluff condition) with a "tight regular who only commits with real hands" (value condition).

\subsubsection{Context Swapping Control Condition}

To distinguish Theory of Mind reasoning from poker-specific pattern matching, we implemented a context swapping control:

\begin{itemize}
    \item \textbf{Swapped Stimuli}: For each scenario pair, we created variants where opponent psychological descriptions were exchanged while maintaining identical poker mechanics.
    
    \item \textbf{Critical Test}: Models using genuine Theory of Mind should change predictions when opponent psychology changes, while models relying on poker heuristics should maintain consistent predictions.
    
    \item \textbf{Quantification}: We measured context sensitivity as the percentage change in predictions between original and swapped scenarios, providing a direct measure of Theory of Mind reliance.
\end{itemize}

This control addresses the fundamental validity question: Are models reasoning about mental states or merely applying poker-specific patterns?

\subsection{Model Selection and Evaluation}
\label{subsec:model_evaluation}

\subsubsection{Language Models}

We evaluated four state-of-the-art LLMs representing different architectures and training approaches:

\begin{itemize}
    \item \textbf{Hush-Qwen2.5-7B}: A 7B parameter model fine-tuned for instruction following and reasoning tasks
    \item \textbf{OLMoE-1B-7B-0125-Instruct}: A mixture-of-experts model with 1B active parameters
    \item \textbf{EXAONE-3.5-2.4B-Instruct}: A 2.4B parameter model optimized for mathematical and logical reasoning  
    \item \textbf{Llama-3.2-SUN-HDIC-1B-Ins}: A compact 1B parameter variant of the Llama architecture
\end{itemize}

This selection provides coverage across model sizes, architectures, and training methodologies, enabling robust conclusions about Theory of Mind capabilities across the current LLM landscape.

\subsubsection{Evaluation Protocol}

For each stimulus, models received the complete scenario description and were prompted to:

\begin{enumerate}
    \item Classify the opponent's bet as BLUFF or VALUE
    \item Provide detailed reasoning for their decision
    \item Explain their assessment of the opponent's mental state and intentions
\end{enumerate}

We used a zero-shot evaluation paradigm to assess models' inherent Theory of Mind capabilities without task-specific fine-tuning. Each model processed all 30 scenarios in randomized order to control for sequence effects.

\subsubsection{Response Analysis}

Model responses underwent both quantitative and qualitative analysis:

\textbf{Quantitative Metrics:}
\begin{itemize}
    \item Overall classification accuracy
    \item Bluff vs. value detection accuracy
    \item Context sensitivity (percentage change in swapped scenarios)
    \item Statistical significance testing (Fisher's exact test, Cohen's d)
\end{itemize}

\textbf{Qualitative Analysis:}
\begin{itemize}
    \item Mental state attribution frequency and sophistication
    \item Recursive reasoning patterns ("I think that he thinks...")
    \item Belief attribution and intention recognition
    \item Cognitive complexity categorization
\end{itemize}

\subsection{Theory of Mind Assessment Framework}
\label{subsec:tom_framework}

Our evaluation framework draws from developmental psychology research on Theory of Mind while adapting to the linguistic and strategic complexities of our task.

\subsubsection{Theory of Mind Indicators}

We identified four key markers of Theory of Mind reasoning in model responses:

\begin{enumerate}
    \item \textbf{Belief Attribution}: Explicit reasoning about what the opponent believes about their hand strength, position, or situation
    
    \item \textbf{Intention Recognition}: Understanding opponent goals and strategic intentions beyond immediate actions
    
    \item \textbf{Recursive Reasoning}: Multi-level mental state reasoning ("He knows that I know...")
    
    \item \textbf{Contextual Integration}: Synthesis of opponent psychology with situational factors
\end{enumerate}

\subsubsection{Coding Schema}

Two independent coders analyzed all model responses using a structured coding schema:

\begin{itemize}
    \item \textbf{Mental State Attribution} (Present/Absent): Does the response explicitly reference opponent mental states?
    
    \item \textbf{Recursive Depth} (0-3 levels): How many levels of recursive reasoning are demonstrated?
    
    \item \textbf{Psychological Sophistication} (Low/Medium/High): Complexity and accuracy of psychological analysis
    
    \item \textbf{Context Integration} (Poor/Good/Excellent): How well does the model integrate opponent psychology with poker mechanics?
\end{itemize}

Inter-rater reliability was high across all categories (Cohen's κ > 0.85), confirming the reliability of our qualitative assessments.

\subsection{Statistical Analysis}
\label{subsec:statistical_analysis}

\subsubsection{Performance Comparisons}

We used Fisher's exact tests to compare classification accuracy between models, given the categorical nature of our dependent variable and relatively small sample sizes. Effect sizes were calculated using Cohen's d for continuous measures and Cramér's V for categorical comparisons.

\subsubsection{Context Sensitivity Analysis}

Context sensitivity was quantified as:

\begin{equation}
Context\ Sensitivity = \frac{|Accuracy_{original} - Accuracy_{swapped}|}{Accuracy_{original}} \times 100\%
\end{equation}

We used paired t-tests to compare original versus swapped scenario performance within each model, and independent t-tests to compare context sensitivity between models.

\subsubsection{Robustness Checks}

To ensure the reliability of our findings, we conducted several robustness analyses:

\begin{itemize}
    \item \textbf{Bootstrap Resampling}: 1000 bootstrap samples to estimate confidence intervals for all accuracy measures
    
    \item \textbf{Cross-Validation}: Leave-one-out cross-validation to assess generalization across scenarios
    
    \item \textbf{Sensitivity Analysis}: Examination of results under different coding thresholds and classification criteria
\end{itemize}

These analyses confirm that our results are robust to methodological choices and represent genuine model differences rather than experimental artifacts. 